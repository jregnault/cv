\documentclass[12pt, a4paper]{article}

% Police
\usepackage{arev}

\usepackage[explicit]{titlesec}
\usepackage{titling}
\usepackage[margin=0.75cm,top=0cm]{geometry}
\usepackage{color}
\usepackage{tcolorbox}
\usepackage{graphicx}
\usepackage{multicol}

\definecolor{darkblue}{rgb}{0.10,0.45,0.85}

% ==================== Informations ==================== %
\author{Jérémie Regnault}
\title{Ingénieur logiciel}
\newcommand{\address}{59320 Hallennes-Lez-Haubourdin}
\newcommand{\phone}{06.41.93.47.26}
\newcommand{\email}{jeremie.regnault@outlook.com}
\newcommand{\webpage}{jregnault.github.io}
\newcommand{\mobility}{Mobilité sur Lille et alentours}

% ==================== Formatting ====================== %
\renewcommand{\maketitle}{
	\begin{tcolorbox}[
		size=tight,oversize,
		sharp corners,
		colback=darkblue,
		colframe=cyan,
		left=56pt,
		right=90pt,
		top=0.75cm,
		bottom=0.75cm
	]
    \begin{minipage}[t]{.15\textwidth}
        \includegraphics[width=\textwidth]{portrait.png}
    \end{minipage}
    \hfill
    \begin{minipage}[t]{.80\textwidth}
        \vspace{-2.7cm}
        \begin{multicols}{2}
            \par{\Huge \textcolor{white}{\theauthor}}
            \par{\LARGE \textcolor{white}{\thetitle} \par}
            \hfill
            \begin{flushright}
                \par{
                    \textcolor{white}{
                        \address \\
                        \phone \\
                        \email \\
                        \webpage \\
                        \mobility \\
                    }
                }
            \end{flushright}
        \end{multicols}
    \end{minipage}
\end{tcolorbox}
}

\titleformat{\section}
    {\Large}
    {\textcolor{darkblue}{#1}}
    {0em}
    {}
    [\titlerule]

\newcommand{\cventry}[4]{
    \begin{minipage}[t]{2.5cm}
        \centering{#1}
    \end{minipage}
    \vline
    \hspace{.5cm}
    \begin{minipage}[t]{.85\linewidth}
        \textcolor{darkblue}{\large \bf#2}\\\textit{#3} \\\footnotesize{#4}
        \vspace{.25cm}
    \end{minipage}
}

\newcommand{\cvinterest}[4]{
    \begin{minipage}[t]{2.5cm}
        \centering{#1}
    \end{minipage}
    \hspace{.5cm}
    \begin{minipage}[t]{.85\linewidth}
        {\large \bf#2}\\\textit{#3} \\\footnotesize{#4}
        \vspace{.25cm}
    \end{minipage}
}

\newcommand{\cvitem}[2]{
    \begin{minipage}[t]{2.5cm}
        {\bf#1} \hfill
    \end{minipage}
    #2
    \hfill\\
}

\begin{document}
    \maketitle
    \vspace*{-0.5cm}

    \section{Compétences}
    \begin{minipage}[t]{.28\linewidth}
        \textcolor{darkblue}{\large \bf Langages} \\
        $\bullet$ Python 3 \\
        $\bullet$ HTML5 \\
        $\bullet$ Java
    \end{minipage}
    \begin{minipage}[t]{.28\linewidth}
        \textcolor{darkblue}{\large \bf Technologies} \\
        $\bullet$ Git \\
        $\bullet$ Linux \\
        $\bullet$ Maven
    \end{minipage}
    \begin{minipage}[t]{.5\linewidth}
        \textcolor{darkblue}{\large \bf Savoir-faire} \\
        $\bullet$ Élaborer un cahier des charges \\
        $\bullet$ Méthode AGILE \\
        $\bullet$ Conception / développement applicatif
    \end{minipage}

    \vspace*{-0.5cm}

    \section{Expériences}
        \cventry{Mars -- Août 2020}{Ingénieur Intelligence Artificielle (Stagiaire)}{Elosi, Villeneuve d'Ascq}{
            $\bullet$ Montée en compétence en traitement du langage naturel (\textbf{NLP}) \\
            $\bullet$ Montée en compétence dans le deep learning, plus spécifiquement les modèles encodeurs / décodeurs \\
            (\textbf{Python 3}, \textbf{TensorFlow}, \textbf{PyTorch})
        }
        \cventry{Juin -- Août 2017}{Analyste Programmeur (Stagiaire)}{Inria - Nord Europe - Équipe RMoD}{
            $\bullet$ Analyse des problèmes rencontrés lors de l'utilisation de Pharo dans un environnement en lecture seule \\
            $\bullet$ Réalisation d'une étude auprès de la communauté Pharo pour définir les axes d'amélioration du langage \\
            (\textbf{Pharo})
        }
    
    \vspace*{-0.5cm}

    \section{Formation}
        \cventry{2017 \\--\\ 2020}{Master Informatique}{Université de Lille}{
            Spécialisation en Modèles Complexes, Algorithmes et Données (MoCAD)\\
            $\bullet$ Systèmes multi-agents simples (simulation de particules, wator)(\textbf{Python 3}, \textbf{PyGame})\\
            $\bullet$ Algorithme d'optimisation combinatoire basé sur des heuristiques pour résoudre le problème du voyageur de commerce (\textbf{Python 3})\\
            $\bullet$ Interpréteur JavaScript basique (\textbf{Java}, \textbf{C}) \\
            $\bullet$ Gestionnaire de bibliothèque (\textbf{Java EE}) \\
            $\bullet$ Serveur FTP avec API REST (\textbf{Java})
        }
        \cventry{2012 \\--\\ 2017}{Licence Informatique}{Université de Lille}{}

    \vspace*{-0.5cm}

    \section{Langues}
        \cvitem{Anglais}{Courant (Niveau C2, TOEIC: 985)}
    
    \vspace*{-0.5cm}

    \section{Centres d'intérêt}
        \cvinterest{depuis 2016}{Escrime Artistique}{Académie d'Escrime Vauban Lille (AEVL)}{
            Création de combats chorégraphiés à destination de pièces de théâtres et de scénettes.
        }
        
\end{document}